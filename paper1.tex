\documentclass[a4paper]{article}
\usepackage{graphicx}
\usepackage{onecolceurws}

\title{Sophize Markdown and Communication Interface }

\author{
Abhishek Chugh \\ Sophize Foundation\\
                Bengaluru, India \\ abc@sophize.org
}

\institution{}




\begin{document}
\maketitle

\begin{abstract}
Providing mathematicians to see connections of Mathematical objects (definitions, propositions, proofs) within and across multiple document has the potential to vastly accelerate mathematical research. We extend the markdown language and to demonstrate a new interactive way of representing such deeply connected mathematical knowledge on the web. We also utilize this new language to create a novel communication system that is specially built to allow a large number of collaborators to collaboratively solve mathematical problems.
\end{abstract}
\vskip 32pt


\section{Introduction}

Sophize is a novel Mathematics library and discussion platform. Mathematical proofs are developed from multiple POV (foundations, verification standards) and a single proof spans across a large number of documents - theorems to axioms. Our primary mission is to make it easy for mathematicians to allow mathematicians to find, discuss and discover such proofs. To achieve this mission, we are working on three goals:

formalize the largely unexplored network of information embedded in the connections of mathematical objects as envisioned by the GDML committee.
Using the network of information to combine proofs based on each user’s chosen foundations and verification standards.
Create a platform that is tailored for communication of mathematical ideas which utilizes and augments the network of information.

To achieve the above goals, we believe it is necessary to create a new format that is as extremely convenient for a casual web conversation of mathematical ideas and is powerful enough to embed mathematical entities (definitions, theorems, proofs etc.) from a wide variety of sources. The markdown language has the simplicity, extensibility and wide acceptance on the web that we are looking for. Sophize markdown extends the markdown format to meet our requirements.

The communication interface on the Sophize is designed to help mathematicians easily make sense of the discussion even when a post receives a large number of comments from many collaborators. We studied the nature of collaboration on Polymath projects - massively collaborative online mathematical projects to come with the design of our interface. Conveniently, the leaders have publically analysed these projects and suggested technical improvements suitable for such collaborations. In fact, a large portion of the interface is derived directly from the list of features desired and posted on their respective blogs by Prof Timothy Gowers and Prof Terence Tao. We believe that these features will not only aid large scale collaborations like Polymath but will also be useful for private collaborations and workshops like AiM.


\section{Preliminaries}

Describe def, prop, arg, resourcepointer.

\subsection{Sophize Markdown}

What is markdown, out of box features

\subsubsection{Latex extension}

Third level headings must be flush left, initial caps and bold.
One line space before the third level heading and $1/2$ line
space after the third level heading.

\subsubsection{Link extension}

Fourth level headings must be flush left, initial caps and roman type.
One line space before the fourth level heading and $1/2$ line
space after the fourth level heading.

\paragraph{Link Types}
Overlay, NO LINK, 

\paragraph{Status indicators}


\paragraph{Resource Expansion}

\subsubsection{Formal Language Support}

Indicate footnotes with a number\footnote{This is a sample footnote} in
the text. Place the footnotes at the bottom of the page they appear on.
Precede the footnote with a vertical rule of 2 inches (12 picas).

\subsubsection{Renderer}
\paragraph{ Indicators }
\paragraph{ Resource Overlay }
\paragraph{ Proof views }

\section{Sophize Collaboration Interface}
Guiding principles/motivations?
\subsection{Hierarchical Page Organization}

\subsection{Using existing math content}

\subsection{Referencing Comments}

\subsection{Moderation Tools}

\subsection{Hierarchical organization of comments}

\subsection{Comment Summaries}

\subsection{Live Latex Preview}

\subsubsection*{Acknowledgements}


%\bibliographystyle{alpha} 
%\bibliography{samplebib}
%inline the .bbl file directly for mailing to authors.

\begin{thebibliography}{Com79}

\bibitem[Com79]{Comer-btree}
D.~Comer.
\newblock The ubiquitous b-tree.
\newblock {\em Computing Surveys}, 11(2):121--137, June 1979.

\bibitem[Knu73]{Knuth-vol3}
D.~E. Knuth.
\newblock {\em The Art of Computer Programming -- Volume 3 / Sorting and
  Searching}.
\newblock Addison-Wesley, 1973.

\end{thebibliography}

\end{document}


